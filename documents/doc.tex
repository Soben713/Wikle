\documentclass[11pt,a4paper]{article}
\usepackage{tikz}
\usepackage{solutions}


\شروع{نوشتار}
\سربرگ[سهیل به‌نژاد ($90109903$) و مهسا درخشان ($90109766$)]{بازیابی پیشرفته‌ی اطلاعات}{مستندات فاز سوم پروژه}{۲۹ دی ۱۳۹۲}

\قسمت{خزش}
\زیرقسمت{محتوای تولید شده سمت کاربر}
برای \لر{index} کردن محتویات تولید شده توسط جاوااسکریپت می‌بایست اجازه دهیم کدهای جاوااسکریپت به طور کامل روی صفحه اعمال شوند و احیاناً اگر درخواست \لر{ajax}ی نیاز بود فرستاده شود. برای این منظور می‌توان از کتابخانه‌هایی که کار مرورگر را شبیه‌سازی می‌کنند مانند \کد{HtmlUnit} استفاده کرد. این کار در کد ما انجام شده است. برای فعال کردن جاوااسکریپت کافی‌ست در کلاس \کد{Crawler} مقدار متغیر \کد{EXECUTE\_JS} را برابر با \کد{true} قرار دهید. \\

البته عمل خزش زمانی که کدهای جاوااسکریپت اجرا می‌شوند بسیار کند می‌شود.
\پایان{نوشتار}
